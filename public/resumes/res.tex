%%%%%%%%%%%%%%%%%%%%%%%%%%%%%%%%%%%%%%%%%%%%%%%%%%%%%%%%%%%%%%%%%%%%%%%%%%%%%%%%
% Medium Length Graduate Curriculum Vitae
% LaTeX Template
% Version 1.2 (3/28/15)
%
% This template has been downloaded from:
% http://www.LaTeXTemplates.com
%
% Original author:
% Rensselaer Polytechnic Institute 
% (http://www.rpi.edu/dept/arc/training/latex/resumes/)
%
% Modified by:
% Daniel L Marks <xleafr@gmail.com> 3/28/2015
%
% Important note:
% This template requires the res.cls file to be in the same directory as the
% .tex file. The res.cls file provides the resume style used for structuring the
% document.
%
%%%%%%%%%%%%%%%%%%%%%%%%%%%%%%%%%%%%%%%%%%%%%%%%%%%%%%%%%%%%%%%%%%%%%%%%%%%%%%%%

%-------------------------------------------------------------------------------
%	PACKAGES AND OTHER DOCUMENT CONFIGURATIONS
%-------------------------------------------------------------------------------

%%%%%%%%%%%%%%%%%%%%%%%%%%%%%%%%%%%%%%%%%%%%%%%%%%%%%%%%%%%%%%%%%%%%%%%%%%%%%%%%
% You can have multiple style options the legal options ones are:
%
%   centered:	the name and address are centered at the top of the page 
%				(default)
%
%   line:		the name is the left with a horizontal line then the address to
%				the right
%
%   overlapped:	the section titles overlap the body text (default)
%
%   margin:		the section titles are to the left of the body text
%		
%   11pt:		use 11 point fonts instead of 10 point fonts
%
%   12pt:		use 12 point fonts instead of 10 point fonts
%
%%%%%%%%%%%%%%%%%%%%%%%%%%%%%%%%%%%%%%%%%%%%%%%%%%%%%%%%%%%%%%%%%%%%%%%%%%%%%%%%

\documentclass[margin]{res}  

% Default font is the helvetica postscript font
\usepackage{helvet}

% Increase text height
\textheight=700pt

\begin{document}

%-------------------------------------------------------------------------------
%	NAME AND ADDRESS SECTION
%-------------------------------------------------------------------------------
\name{Harry Bendekgey}

% Note that addresses can be used for other contact information:
% -phone numbers
% -email addresses
% -linked-in profile

\address{}
\address{www.hbendekgey.me\\github.com/hbendekgey\\linkedin.com/in/hbendekgey}

% Uncomment to add a third address
%\address{Address 3 line 1\\Address 3 line 2\\Address 3 line 3}
%-------------------------------------------------------------------------------

\begin{resume}

%-------------------------------------------------------------------------------
%	EDUCATION SECTION
%-------------------------------------------------------------------------------
\section{Education}
\textbf{Ph.D. Candidate, University of California, Irvine} \hfill Fall 2019-present\\
Advised by Erik Sudderth. Advanced to Candidacy Spring 2022.\hfill GPA: 4.0\\ \\
\textbf{B.A., Pomona College} \hfill Graduated Spring 2019\\
Computer Science and Mathematics (Double Major) \hfill GPA: 3.97\\
Thesis: Clustering Player Strategies from Variable-Length Game Logs in Dominion
%-------------------------------------------------------------------------------

%-------------------------------------------------------------------------------
%	EXPERIENCE SECTION
%-------------------------------------------------------------------------------
% Modify the format of each position
\begin{format}
\title{l}\dates{r}\\
\employer{l}\location{r}\\
\body\\
\end{format}
%-------------------------------------------------------------------------------

%\employer{Computer Science Department}
%\location{Pomona College}
%\dates{Fall 2018}
%\title{\textbf{Research Fellow}}
%\begin{position}
%Designed a novel approach for understanding the landscape of personas across games;\\
%Scraped and cleaned data from online \emph{Dominion} logs to visualize player strategies.
%\end{position}
%
%\employer{Kenneth Cooke Research Fellowship}
%\location{Pomona College}
%\dates{Summer 2018}
%\title{\textbf{Research Fellow}}
%\begin{position}
%Researched the mathematical models underlying state-of-the-art election forecasting;\\
%Implemented adjustable forecasts for the 2018 midterm election based on these models.
%\end{position}

\section{Teaching Experience}
\employer{ICS 6N: Computational Linear Algebra}
\location{University of California, Irvine}
\dates{Summer 2023}
\title{\textbf{Instructor of Record}}
\begin{position}
	For the 10-week summer session: developed all course materials, lectured, and graded
\end{position}

\employer{CS 177: Applications of Probability in CS}
\location{University of California, Irvine}
\dates{Fall 2020, Winter 2024, Fall 2024}
\title{\textbf{Teaching Assistant}}
\begin{position}
	Designed exam content in addition to standard TA duties: \\Holding office hours and discussion sections; grading assignments.
\end{position}

\employer{Computer Science Department}
\location{Pomona College}
\dates{}
\title{\textbf{Teaching Assistant}}
\begin{position}
$\circ$ Discrete Math and Functional Programming (Head TA)\hfill Spring 2019\\
$\circ$ Computer Systems \hfill Fall 2018\\
$\circ$ Fundamentals of Computer Science (Head TA) \hfill Spring 2017, Spring 2018\\
$\circ$ Introduction to Computer Science \hfill Spring 2016-Fall 2016

\end{position}

\employer{Mathematics Department}
\location{Pomona College}
\dates{}
\title{\textbf{Mentor and Grader}}
\begin{position}
$\circ$ Combinatorial Mathematics \hfill Spring 2017, Spring 2018, Spring 2019\\
$\circ$ Linear Algebra \hfill Fall 2018\\
$\circ$ Statistical Theory \hfill Fall 2018

\end{position}

\section{Industry Experience}

\employer{Chan Zuckerberg Biohub}
\location{San Francisco}
\dates{Summer 2019}
\title{\textbf{Intern}}
\begin{position}
Work with the theory group on two projects touching biology, physics, and statistics:\\
$\circ$ Investigating the ability of (MC)$^3$ to explore the space of phylogenetic trees, and \\
$\circ$ Discovering a new power law for modeling diffusion in crowded dynamic spaces
\end{position}

\employer{QuanticMind}
\location{}
\dates{Summer 2017}
\title{\textbf{Engineering Intern}}
\begin{position}
Created an API for employees to access databases without requiring access credentials;
Led meetings with colleagues to generate common use cases to be addressed by API.\end{position}

%-------------------------------------------------------------------------------

%-------------------------------------------------------------------------------
%	PUBLICATION SECTION
%-------------------------------------------------------------------------------

\section{Publications}
\par
Unbiased learning of deep generative models with structured discrete \\representations. \\ H Bendekgey, G Hope, E Sudderth. NeurIPS 2023

Scaling Study of Diffusion in Dynamic Crowded Spaces. \\H Bendekgey, G Huber, D Yllanes, L Yan. APS March Meeting 2022

Scalable \& Stable Surrogates for Flexible Classifiers with Fairness Constraints. \\ H Bendekgey, E Sudderth. NeurIPS 2021

Clustering Player Strategies from Variable-Length Game Logs in \emph{Dominion}.\\ H Bendekgey, AAAI Workshop on Knowledge Extraction from Games (KEG), 2019.

\section{Publications\\ In Submission}

%Human Motion Modeling with Discretized Dynamics. \\ {H Bendekgey}, G Hope, E Sudderth. 

Fast inference in sparse time series. \\ {H Bendekgey}, D Sujono, M Motamed, E Sudderth. 

Undergraduate data science education: Who has the microphone and what are they saying?\\
M Dogucu, S Demirci, H Bendekgey, FZ Ricci, CM Medina. 

%Consistency and Reproducibility in U.S. House of Representatives Forecasts.\\ \textbf{H Bendekgey}, arXiv preprint arXiv:1811.12466, 2018

%-------------------------------------------------------------------------------

%-------------------------------------------------------------------------------
%	COMPUTER SKILLS SECTION
%-------------------------------------------------------------------------------

%\section{Projects}
%\textbf{Midterms Forecasting Website}\\
%github.com/hbendekgey/midterms-website\\
%Codebase for a fully interactive website to understand the differences in popular election forecasting methods, and to see how changing assumptions or parameters affects the final forecast.
%
%\textbf{Spotify Data Science Workshop}\\
%github.com/hbendekgey/Spotify-Workshop\\
%Detailed step-by-step instructions for how to use Spotify's API to get audio features of one's own music library, and then practice common machine learning techniques in Python on that data.
%
%\par
%\textbf{Cellular Automata}\\ 
%varsncrafts.com/\#/crafts/react/Cellular\%20Automaton\\
%Implementation of a generic cellular automaton with modifiable rule space, to understand how emergent properties can manifest in seemingly simple systems.

 

%-------------------------------------------------------------------------------
%	Interests
%-------------------------------------------------------------------------------
\section{Talks}

Why We Use Reverse-Mode Autodiff (And the Time I Didn't) \hfill Feb 2024\\ 
UC Irvine DataLab Seminar

Unbiased Learning of Deep Generative Mdels with Discrete Representations \hfill Nov 2023\\ 
Pomona College Computer Science Colloquium Series.

\section{Department Service}



AI Faculty Search Committee, Student Member \hfill 2022-2024\\
I was part of a group of 4-6 PhD students who interviewed faculty candidates with a focus on their research, their advising styles, and their interactions with PhD students. 

HPI@UCI Reading Group Organizer \hfill 2021-2022\\
Organized a cross-lab reading group of 15 student fellows across machine learning specializations for the 2021-2022 academic year.

HPI@UCI Workshop Organizer \hfill 2021-2022\\
Coordinated talks and activities for 30 workshop attendees from UC Irvine and the Hasso-Plattner Institute in Germany.

\section{UC Irvine\,\, Awards}

Hasso Plattner Institute Fellowship \hfill 2021-2023\\
Provides 3 years of full Ph.D funding for work on adaptive, safe and human-centered artificial intelligence.

Enhanced Computer Science Department Excellence Fellowship \hfill 2019-2020\\
Allows first-year Ph.D students to engage sooner and more deeply with research by dispensing with teaching assistant requirement.

Dean's Award \hfill 2019\\
Extra first-year grant for outstanding research potential.


%\section{Pomona College Awards}
%
%\textbf{Paul B. Yale Computer Science Prize} \hfill 2019\\
%Awarded annually to an outstanding senior majoring in Computer Science.
%
%\textbf{Phi Beta Kappa Award} \hfill 2019\\
%Awarded to a senior for high quality of scholarship and promise of future distinction.
%
%\textbf{Kenneth Cooke Summer Research Fellowship} \hfill 2018\\
%Grant for summer research in an area of applied mathematics or statistics.
%
%\textbf{Bruce Jay Levy Prize in Mathematics} \hfill 2018\\
%Awarded annually to a student for excellence in the field of mathematics.
%
%\textbf{Llewellyn Bixby Mathematics Prize} \hfill 2017\\
%Awarded annually to a sophomore for excellence in the second year of mathematics.
%
%\section{National \\Awards}
%\textbf{Phi Beta Kappa Member} \hfill Elected Junior year, 2018\\
%The oldest honor society in the country; at eligible schools, 2\% of Juniors are elected.
%
%\textbf{National Merit Scholar} \hfill 2015\\
%College scholarship based on performance on the Practice SAT.
%
%\textbf{Caroline D. Bradley Scholar} \hfill 2011-2015\\
%Merit-based, four-year high school scholarship granted to 11 students nationally.

%-------------------------------------------------------------------------------
\end{resume}
\end{document}